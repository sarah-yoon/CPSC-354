\documentclass{article}

\usepackage{tikz} 
\usetikzlibrary{automata, positioning, arrows} 

\usepackage{amsthm}
\usepackage{amsfonts}
\usepackage{amsmath}
\usepackage{amssymb}
\usepackage{fullpage}
\usepackage{color}
\usepackage{parskip}
\usepackage{hyperref}
  \hypersetup{
    colorlinks = true,
    urlcolor = blue,       % color of external links using \href
    linkcolor= blue,       % color of internal links 
    citecolor= blue,       % color of links to bibliography
    filecolor= blue,        % color of file links
    }
    
\usepackage{listings}

\definecolor{dkgreen}{rgb}{0,0.6,0}
\definecolor{gray}{rgb}{0.5,0.5,0.5}
\definecolor{mauve}{rgb}{0.58,0,0.82}

\lstset{frame=tb,
  language=haskell,
  aboveskip=3mm,
  belowskip=3mm,
  showstringspaces=false,
  columns=flexible,
  basicstyle={\small\ttfamily},
  numbers=none,
  numberstyle=\tiny\color{gray},
  keywordstyle=\color{blue},
  commentstyle=\color{dkgreen},
  stringstyle=\color{mauve},
  breaklines=true,
  breakatwhitespace=true,
  tabsize=3
}

\newtheoremstyle{theorem}
  {\topsep}   % ABOVESPACE
  {\topsep}   % BELOWSPACE
  {\itshape\/}  % BODYFONT
  {0pt}       % INDENT (empty value is the same as 0pt)
  {\bfseries} % HEADFONT
  {.}         % HEADPUNCT
  {5pt plus 1pt minus 1pt} % HEADSPACE
  {}          % CUSTOM-HEAD-SPEC
\theoremstyle{theorem} 
   \newtheorem{theorem}{Theorem}[section]
   \newtheorem{corollary}[theorem]{Corollary}
   \newtheorem{lemma}[theorem]{Lemma}
   \newtheorem{proposition}[theorem]{Proposition}
\theoremstyle{definition}
   \newtheorem{definition}[theorem]{Definition}
   \newtheorem{example}[theorem]{Example}
\theoremstyle{remark}    
  \newtheorem{remark}[theorem]{Remark}

\title{CPSC-354 Report}
\author{Sarah Yoon  \\ Chapman University}

\date{\today} 

\begin{document}

\maketitle

\begin{abstract}

\end{abstract}

\setcounter{tocdepth}{3}
\tableofcontents

\section{Introduction}\label{intro}

\section{Week by Week}\label{homework}

\subsection{Week 1}

\subsubsection*{Notes}
Learned about some tactics and theorems

rfl: a tactic that proves theorems that take the form of X = X

rw: a tactic that rewrites a proof

one\_eq\_succ\_zero: a theorem that proves 1 = succ 0 (there are also other similar existing theorems like two\_eq\_succ\_one and so on)

add\_zero: a theorem that proves a + 0 = a.

add\_succ: a theorem that proves a + succ b = succ(a + b)

succ\_eq\_add\_one: a theorem that proves succ a = a + 1

\subsubsection*{Homework}
Problem 5: \\
a b c are in the set of natural numbers. \\
Prove that both sides are equal to each other. \\
a + (b + 0) + (c + 0) = a + b + c \\
rw [add\_zero] - uses the add\_zero theorem to prove that b + 0 = b \\
This is rewritten to: \\
a + b + (c + 0) = a + b + c \\
rw [add\_zero] - this is done again to prove that c + 0 = c \\ 
This is rewritten to: \\
a + b + c = a + b + c \\
rfl - this proves that both sides that look the same are equal to each other \\

Problem 6: \\
This is the same problem as 5 but will be approached in a different manner. \\
a + (b + 0) + (c + 0) = a + b + c \\
rw [add\_zero c] - specifically applies the add\_zero theorem to c, making c + 0 into c \\
This is rewritten to: \\ 
a + (b + 0) + c = a + b + c \\
rw [add\_zero b] - specifically applies the add\_zero theorem to b, making b + 0 into b \\
This is rewritten to: \\ 
a + b + c = a + b + c \\
rfl - this proves that both sides that look the same are equal to each other \\

Problem 7: \\
n is in the set of natural numbers. \\
Prove that both sides are equal to each other. \\
succ n = n + 1
rw[one\_eq\_succ\_zero] - rewrite 1 into successor 0
This is rewritten to: \\ 
succ n = n + succ 0 \\
rw[add\_succ] - uses the add\_succ theorem to change n + succ 0 into succ(n + 0) \\
This is rewritten to: \\ 
succ n = succ (n + 0) \\
rw[add\_zero] - uses the add\_zero theorem to prove that n + 0 = n \\
This is rewritten to: \\ 
succ n = succ n \\
rfl - this proves that both sides that look the same are equal to each other \\

Problem 8: \\
Prove that both sides are equal to each other. \\
2 + 2 = 4 \\
rw[two\_eq\_succ\_one] - rewrites 2 into succ 1 \\
This is rewritten to: \\ 
succ 1 + succ 1 = 4 \\
rw[one\_eq\_succ\_zero] - rewrites 1 into succ 0 \\
This is rewritten to: \\ 
succ (succ 0) + succ (succ 0) = 4 \\
rw[four\_eq\_succ\_three] - rewrites 4 into succ 3 \\
This is rewritten to: \\ 
succ (succ 0) + succ (succ 0) = succ 3 \\
rw[three\_eq\_succ\_two] - rewrites 3 into succ 2 \\
This is rewritten to: \\ 
succ (succ 0) + succ (succ 0) = succ (succ 2) \\
rw[two\_eq\_succ\_one] - rewrites 2 into succ 1 \\
This is rewritten to: \\ 
succ (succ 0) + succ (succ 0) = succ (succ (succ 1)) \\
rw[one\_eq\_succ\_zero] - rewrites 1 into succ 0 \\
This is rewritten to: \\ 
succ (succ 0) + succ (succ 0) = succ (succ (succ (succ 0))) \\
rw[add\_succ] - changes succ (succ 0) + succ (succ 0) into succ (succ (succ 0) + succ 0) \\
This is rewritten to: \\ 
succ (succ (succ 0) + succ 0) = succ (succ (succ (succ 0))) \\
rw[add\_succ] - changes succ (succ (succ 0) + succ 0) into succ (succ (succ (succ 0) + 0)) \\
This is rewritten to: \\ 
succ (succ (succ (succ 0) + 0)) = succ (succ (succ (succ 0))) \\
rw[add\_zero] - changes succ (succ 0) + 0 into \\
This is rewritten to: \\ 
succ (succ (succ (succ 0))) = succ (succ (succ (succ 0))) \\ 
rfl - this proves that both sides that look the same are equal to each other \\

For level 5: 
add\_zero is a Lean proof that a + 0 = a (a representing any number). In mathematics, there are laws for arithemic. One of them 
is called the identity which applies to addition and multiplication. For addition, it states that m + 0 = m = 0 + m. This is the
exact same as the Lean proof, a + 0 = a, which can also be written as a = 0 + a.

%In case you want to draw automata in Latex, you can use the tikz %package. Here is an example of a simple automaton:
%
%\begin{tikzpicture}[shorten >=1pt,node distance=2cm,on grid,auto] 
%  \node[state] (q_1)   {$q_1$}; 
%  \node[state] (q_2) [above right=of q_1] {$q_2$}; 
%  \node[state] (q_3) [below right=of q_2] {$q_3$}; 
%   \path[->] 
%   (q_1) edge  node {0} (q_2)
%         edge  node [swap] {1} (q_3)
%   (q_2) edge  node  {1} (q_3)
%         edge [loop above] node {0} ()
%   (q_3) edge [loop below] node {0,1} ();
%\end{tikzpicture}
%
%By the way, GPT-4 is quite good at outputting tikz code.

\subsubsection*{Comments and Questions}
Learning the root of mathematics is very eye-opening, and I am confident it will be the same for programming languages. 
It provides another perspective for elementary functions like 2 + 2 equals 4, which is different from just knowing it through memorization. 
I feel as though this is why people have been able to expand mathematically. This makes me wonder: how can looking through the core of 
programming help us better current languages (e.g. python, rust)?
%I expect you to read the lecture notes. 

\subsection{Week 2}


\subsection{\ldots}

\ldots

\section{Lessons from the Assignments}


\section{Conclusion}\label{conclusion}


\begin{thebibliography}{99}
\bibitem[BLA]{bla} Author, \href{https://en.wikipedia.org/wiki/LaTeX}{Title}, Publisher, Year.
\end{thebibliography}

\end{document}
